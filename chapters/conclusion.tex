\newpage
\begin{center}
  \textbf{\large ЗАКЛЮЧЕНИЕ}
\end{center}
\refstepcounter{chapter}
\addcontentsline{toc}{chapter}{ЗАКЛЮЧЕНИЕ}

Основные результаты, достигнутые в данной работе можно выделить следующим образом:

\begin{enumerate}
  \item Разработана модификация алгоритм PCA-Seq, позволяющая анализировать наборы молекулярных последовательностей.
  \item Проведен анализ различных метрик для вычисления расстояний между фрагментами последовательностей и фазовыми траекториями, исследовано их влияние на результаты работы алгоритма.
  \item Продемонстрирована высокая эффективность предложенного алгоритма на примере задачи предсказания термофильности организмов. Использование нового метода позволило значительно повысить точность предсказаний по сравнению с существующими результатами, что подчеркивает его практическую ценность.
\end{enumerate}

Преимуществами метода являются:

\begin{itemize}
  \item отсутствие необходимости выравнивать исходные последовательности по длине,
  \item отсутствие необходимости дополнительных входных данных, помимо самих последовательностей.
\end{itemize}

Из недостатков можно отметить:

\begin{itemize}
  \item необходимость тонкой настройки параметров (выбор метрики, размера и шага скользящего окна),
  \item высокая вычислительная сложность.
\end{itemize}

В дальнейшем планируется:
\begin{itemize}
  \item оптимизировать алгоритм по времени,
  \item разместить его реализацию в открытом доступе в виде подключаемой библиотеки,
  \item использовать его для предобработки и визуализации данных в других биологических задачах.
\end{itemize}

Актуальная реализация разработанного метода размещена в открытом репозитории на платформе GitHub \cite{github}.
