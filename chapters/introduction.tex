\newpage
%\begin{center}
%  \textbf{\large АННОТАЦИЯ}
%\end{center}

\setcounter{page}{2}

\newpage \renewcommand{\contentsname}{\centerline{\large СОДЕРЖАНИЕ}}
\tableofcontents

\newpage
\begin{center}
  \textbf{\large ВВЕДЕНИЕ}
\end{center}
\addcontentsline{toc}{chapter}{ВВЕДЕНИЕ}

В современной биоинформатике анализ молекулярных последовательностей играет ключевую роль в исследовании различных биологических процессов. Возрастающий объем геномных и протеомных данных требует разработки новых методов и алгоритмов для их обработки и интерпретации.

Алгоритм PCA-Seq \cite{Efimov2020}, разработанный в 2018 году и основанный на методе главных координат (PCoA) \cite{Gower1966}, продемонстрировал новый подход к анализу молекулярных последовательностей и позволил эффективно снижать размерность молекулярных данных, сохраняя при этом важные биологические характеристики.

В работе проведено исследование свойств алгоритма PCA-Seq, а также предложено его расширение, которое позволяет обрабатывать не только отдельные последовательности, но и анализировать и сравнивать их наборы.

Актуальность работы определяется частой необходимостью векторизации наборов нечисловых последовательностей, в частности~--- молекулярных, например, при предобработке данных для моделей машинного обучения.

Новизна метода заключается в том, что он позволяет применять метод главных компонент и получать наглядное геометрическое представление для наборов молекулярных последовательностей, не требуя того, чтобы они были выровнены по длине, а также не используя дополнительный контекст и известные свойства в качестве признаков, а обрабатывая только непосредственно строковую последовательность нуклеотидов или аминокислот.

Целью данной работы была разработка метода анализа и геометрического представления молекулярных последовательностей, основанного на алгоритме PCA-Seq, и проверка его эффективности на разнообразных биологических данных.

Для достижения поставленной цели были определены следующие задачи:

\begin{enumerate}
  \item Реализация алгоритма PCA-Seq, позволяющего анализировать отдельные последовательности.
  \item Сравнение эффективности различных метрик при их использовании в алгоритме PCA-Seq.
  \item Расширение алгоритма PCA-Seq для анализа наборов последовательностей.
  \item Апробация алгоритма на реальных биологических данных, включая задачу предсказания термофильности организмов по аминокислотной последовательности их белков.
  \item Сравнение эффективности предложенного метода с существующими подходами.
\end{enumerate}
